%\oldSymbols % Marvosymbols

%\makeatletter
%%\@setplength{subjectaftervskip}{\baselineskip+5pt} % 29pt default %Abstand Betreff <-> Anrede
%%\@addtoplength{refvpos}{-8mm} %Textbereich 8mm nach oben
%\makeatother 
%\setlength{\footskip}{0pt} % Abstand Anlagen Seitenende anpassen (auch negativ, wie -8pt)
% Worttrennungshilfen
%	\linebreak \mbox{} \sloppy  \showhyphens{}   \hyphenation{} Wort\-trennbeispiel
\emergencystretch=20pt\tolerance=1200\hyphenpenalty=1000% Gewichte für Trennung
%\hyphenation{Louisiana}% nicht trennen
% zeilenumbrueche mit \linebreak
% Worttrennung mit Wort\-trennung erzwingen


\begin{letter}{%
Bayerisches Staatsministerium für Umwelt und Verbraucherschutz\\
Referat 11\\
Rosenkavalierplatz 2\\
81925 München%
}


\setkomavar{subject}[Betreff ]{Bewerbung für das Baureferendariat}
\opening{Sehr geehrte Damen und Herren,}

mein Masterstudium neigt sich dem Ende zu und nach reiflicher Überlegung habe ich beschlossen, mich für eine Laufbahn in der 4. Qualifikationsebene der bayerischen Wasserwirtschaft zu bewerben. Dafür gibt es mehrer Gründe und es gibt auch mehrere Gründe warum ich mich dafür als geeignet betrachte. Neben der Sicherheit, der guten Vereinbarkeit mit einer Familie und der fairen Bezahlung die eine Beamtenstelle bietet, sind auch Heimatverbundenheit und das Interesse Projekte langfristig zu begleiten Punkte, die mich zu einer Bewerbung veranlassen. 

Mit meinem Bachelor in Geoökologie bringe ich ein breites Verständnis für die Zusammenhänge zwischen Hydro-, Atmo-, Geo- und Biosphäre mit.  Ein Erasmussemester in Amsterdam war nicht nur eine persönlich bereichernde Erfahrung, sondern erlaubte auch interessante Einblicke in den niederländischen Umgang mit Wasser. Durch das Zusatzstudium Umweltrecht habe ich bereits Erfahrung im Umgang mit Gesetzen, Verordnungen und technischen Regeln gesammelt. Im Masterstudium Water Science & Engineering mit der Vertiefung Environmental System Dynamics and Management am Karlsruher Institut für Technologie konnte ich viel dazulernen. Speziell die Arbeit mit Skriptsprachen hat es mir dabei angetan. Ich bin im Allgemeinen sehr technikbegeistert und bastle auch im privaten gerne an kleinen Elektronik Projekten. Von Kindesbeinen an war ich stets in Vereinen engagiert. Speziell der Vereinsfussball hat viel zu meiner Teamfähigkeit beigetragen, da dort Menschen aus verschiedensten sozialen Hintergründen miteinander interagieren. 

Ein einwöchiges Kurzpraktikum im WWA Kempten hat mich nun noch einmal davon überzeugt, dass mir die Arbeit am WWA liegen würde. Daher würde ich mich über eine Einladung zu einem persönlichen Gespräch sehr freuen. 
%\vfill %bei weniger als 3 zeilen platz
\newline\newline\newline 
Mit freundlichen Grüßen\\\sig%
\encl{}
%\vspace*{2mm}Anlagen %spart Platz
\end{letter}
